


Recommender systems are widely deployed in various domains e.g. music(iTunes),
movies(Netflix), e-commerce(Amazon) to suggest relevant products to the users
based on their preferences. 

One of the key methods used by the recommender systems is the collaborative
filtering method, which uses the information from the previous transactions of the users 
and generate the recommendations for the users. Generally, the collaborative
filtering consists of two approaches: the neighborhood-based approach and the
latent-factor models. In the neighborhood-based approach, the similarities
between the items or the users are used to compute recommendations. While the
latent-factor models transforms the users and the items to the same
latent space where both of them are comparable to each other, the items closest
to the users are served as the recommendations to the users.

An important requirement of these collaborative filtering method is that the
user should indicate his/her prefrence over a certain number of the items to
generate recommendations for the future consumption. To overcome this drawback
for a new user, most of the services deploying recommender systems elicit users' preferences on
few items which the user may have consumed before. The recommendation quality
improves with the number of the preferences provided by the user on the items.
This process of eliciting the preferences of users on items can be time
consuming as a user has to indicate his/her preferences on initial items one at a time. 

In this work, we present a method which elicit preferences of the user on a set of
items rather than rating those items individually in a sequence. Since, a set
can be composed of multiple items the  user can indicate his preference on a large
number of items by indicating his preference on few sets of items. In this
manner, the time spent by the user indicating his initial preferences on the items is 
reduced significantly before he begins to use the service. 

%TODO: Introduce model in simple words, like non-linear and linear and sigmoid add citatons

We make the following contributions: First, we propose various Learning from Set
models that enables estimation of the user's preference on the idividual items.
Second, we illustrate the effectiveness of these models with experiments on the 
synthetic and the real datasets by comparing their performance with the state-of-the-art 
matrix factorization methods when the user has indicated his/her preference on
the individual items.










