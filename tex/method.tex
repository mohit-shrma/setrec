%!TEX root = paper.tex

To estimate the preferences on individual items from the preference of the user on the
sets, we need to understand how the user rate a given set of items. A user may
rate a set of items by considering each item in the set or may chose to rate
the set of the items based on few but majority of the items in the set.

In the first case, when a user consider all the items in the set before 
assigning a rating to a set then we can safely assume that the user most likely
gives the set an average score of his preferences for all the items that constitutes 
the set. Under this assumption the estimated rating of the user $u$ on a 
set $S$ is given by: 

\begin{equation} \label{avgSetEq}
  \begin{split}
    \tilde{r}^{avg}_{us} &= \frac{1}{|S|} \sum_{i \in \mathcal{S}} r_{u,i},
  \end{split}
\end{equation}

where $\mathcal{S}$ denotes the set containing the items and $r_{u,i}$ is the
preference score of the user $u$ on the item $i$ .

Similarly, when a user considers only a few but the majority of the items in
the set $\mathcal{S}$ then the user's estimated rating on the set $\mathcal{S}$ is given by:

\begin{equation} \label{majSetEq}
  \begin{split}
    \tilde{r}^{maj}_{us} &= \frac{1}{|\mathcal{S}_{maj}|} \sum_{i \in
    \mathcal{S}_{maj}} r_{u,i},
  \end{split}
\end{equation}

where $\mathcal{S}_maj$ contains the majority of the top rated items in the set 
$\mathcal{S}$.


Assuming that the original user-item preference matrix $R$ is low-rank, we can write
the estimated preference of the user $u$ for the item $i$ as follow:

\begin{equation} 
  \begin{split}
    \tilde{r}_{ui} &= p_u^Tq_i, 
  \end{split}
\end{equation}

where, $k$ is the rank of the matrix $R$, $p_u \in \mathcal{R}^k$ and $q_i \in \mathcal{R}^K$ denotes the latent factor of the user $u$ and the item $i$ respectively.  

Following the low-rank assumption, we can rewrite the estimated score of the 
user $u$ for a set $\mathcal{S}$ in equations \ref{avgSetEq} and \ref{majSetEq} as follow:

\begin{equation} \label{avgSetLoEq}
  \begin{split}
    \tilde{r}^{avg}_{us} &= \frac{1}{|S|} \sum_{i \in \mathcal{S}} p_u^Tq_i,
  \end{split}
\end{equation}


\begin{equation} \label{majSetLoEq}
  \begin{split}
    \tilde{r}^{maj}_{us} &= \frac{1}{|\mathcal{S}_{maj}|} \sum_{i \in
    \mathcal{S}_{maj}} p_u^Tq_i,
  \end{split}
\end{equation}


\subsection{Learning from Set Model}
The learning from set model is parameterized by $\Theta=[P, Q]$, where the
matrices $P$ and $Q$ contains the latent factors of the users and the items
respectively. The model parameters are estimated by minimizing the squared error
loss function, given by

%
\begin{equation} \label{eq_rmse}
  \mathcal{L}_{rmse}(\Theta) \equiv \sum_{u \in U} \sum_{\substack{s \in
  \mathcal{R}_{us}}} (\tilde{r}_{us} - r_{us})^2,
\end{equation}
%

%TODO: need to add better connection to majority or average assumption

where $\mathcal{R}_{us}$ contains all the sets preferred by the user $u$,
$r_{us}$ is the original rating and $\tilde{r}_{us}$ is the estimated rating of
the user $u$ on the set $s$. The estimated rating $\tilde{r}_{us}$ can be
computed as per equation \ref{avgSetLoEq} or \ref{majSetLoEq} depending on
whether the average or the majority assumption is used to estimate the rating of 
the user $u$ on the set $s$. 

To control model complexity, we add regularization of the model parameters
thereby leading to an optimization process of the following form:

%
\begin{equation} \label{eq_obj}
  \min_{P, Q} \sum_{u \in U} \sum_{\substack{s \in \mathcal{R}_{us}}} (\tilde{r}_{us} - r_{us})^2  + \lambda (\|P\|_F^2 + \|Q\|_F^2),
\end{equation}
%

where $\lambda$ is the regularization parameter.

%TODO: need to add better connection to majority or average assumption

The optimization problem of equation \ref{eq_obj} is solved by stochastic
gradient descent algorithm for the average assumption and stochastic
sub-gradient descent algorithm for the majority assumption.

%TODO: add pseudo-code algorithm
%TODO: add sub-gradient details and why it is needed


