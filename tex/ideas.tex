\documentclass[11pt]{article}

\usepackage{epsfig}
\usepackage{amssymb}
\usepackage[fleqn]{amsmath}
\usepackage{bm}
\usepackage{amsfonts}
\usepackage{algorithm}
\usepackage{algpseudocode}
\usepackage{varwidth}
\usepackage{threeparttable}
\usepackage{multirow,bigdelim}
\usepackage{booktabs}
\usepackage{dsfont}
\usepackage{graphicx}


\usepackage{setspace}
\usepackage{hyperref}
\usepackage{xspace}

\DeclareMathOperator*{\argmin}{argmin}
\DeclareMathOperator*{\argmax}{argmax}

\begin{document}

\section{Problem}

Given preference of the user $u$ on a set $s$ of items i.e., $r_{us}$,
our goal is to estimate the user's preference
on items in set i.e., $\tilde{r}_{ui}$, where $i \in s$

\section{Derived set latent factor method}
Estimated preference of a set i.e., $\tilde{r}_{us}$ is given by,

\begin{equation}
  \tilde{r}_{us} = \bm{v_u}^T \bm{g_s},
\end{equation}

where $\bm{v_u}$ is the user $u$'s latent factor and $\bm{g_s}$ is the set $s$'s latent
factor. 

The set latent factor $\bm{g_s}$ can be derived using latent factor or features
of items contained in the set $s$. If each item in the set contributes with
equal weight towards rating of set
then $\bm{g_s}$ is given by average of the latent factor of items.

\begin{equation}
  \bm{g_s} = \frac{1}{|s|}\sum_{\substack{i \in s}}\bm{v_i},
\end{equation}

where $\bm{v_i}$ denotes the latent factor of item $i$ in set $s$.

If user gives rating to a set that is equivalent to rating of the most favoured item
in the set $s$ then the latent factor of the set $s$ is given by,

\begin{equation}
  \bm{g_{us}} = \argmax\limits_{v_i} \bm{v_u}^T \bm{v_i}, 
\end{equation}
where $\bm{g_{us}}$ denotes a personalized latent factor of set $s$ for user
user $u$.

Similarly if a user gives rating to a set that is equivalent to the rating of
the least favoured item in the set $s$ then the latent factor of the set $s$ is
given by,

\begin{equation}
  \bm{g_{us}} = \argmin\limits_{v_i} \bm{v_u}^T \bm{v_i}
\end{equation}

If the user is influenced by majority of items in the set $s$ and based on these
items he assigns a rating to the set $s$, then the latent factor of the set $s$
i.e., $\bm{g_{us}}$ can be approximated as average of latent factor of these
favoured majority of items.

A generalized version of latent factor of the set $s$ can be written as
a weighted average of latent factor of items in the set $s$.

\begin{equation}
  \bm{g_{us}} = \sum_{\substack{i \in s}}\bm{w_{ui}}\bm{v_i}, 
\end{equation}
where $w_{ui}$ denotes the weight given by user $u$ to item $i$ in set and all
the weights sum to $1$ i.e., $\sum_{\substack{i \in s}} \bm{w_{ui}} = 1$.

We can use a loss function like square error loss  or ranking loss over set
preferences to learn the user and item latent factors. We can estimate the
preference of the user $u$ on the item $i$ by a scalar product of the user and the
item latent factor i.e., $r_{ui} = v_u^T v_i$.

\section{Derived set latent factor with intrinsic chateristics for set}
In addition to deriving latent factor of the set using items' latent factor, a
set can has its own intrinsic/latent characteristic. Assuming the set latent factor is
determined by average of items' latent factors and an intrinsic characteristic,
then it can be written as,
\begin{equation}
  \bm{g_s} = \frac{1}{|s|}\sum_{\substack{i \in s}}\bm{v_i} + \bm{h_s} 
\end{equation}

where $h_s$ represent intrinsic/latent characteristic of the set $s$. This approach was used
in paper "Recommending user generated item lists, Liu, Xie and Lakshmanan
(Recsys 2014)" for recommendation of sets to users.


\section{Explicit addition of user weighted intrinsic characteristic score}
If we know that user tends to rate a set more if certain intrinsic property of
the set appeals to him e.g., similaity or diversity of the items in the set, then we can add
user weighted score of property to the estimated preference of the set. 

\begin{equation}
  \tilde{r}_{us}  = \bm{v_u}^T\bm{g_s} + \lambda_{u1}sim(s) +
  \lambda_{u2}diversity(s)
\end{equation}



\end{document}




