
\documentclass{sig-alternate-05-2015}
\usepackage{amssymb}
\usepackage{amsmath}
\usepackage{bm} 
\usepackage{xspace}
\usepackage{algorithm}
\usepackage{algpseudocode}
\usepackage{varwidth}
\usepackage{threeparttable}
\usepackage{multirow,bigdelim} 
\usepackage{booktabs}
\usepackage{dsfont}
\usepackage{epsfig}
\usepackage{graphicx}
\usepackage{hyperref}
\usepackage{xspace}
\usepackage{color}
\usepackage{epstopdf}
\usepackage[font=footnotesize]{subfig}
\usepackage{chngpage}
\usepackage{subfig}
\usepackage{setspace}




\begin{document}

\title{\Large Learning from Sets}
\author{Mohit Sharma~\thanks{University of Minnesota.}}
\date{}
\maketitle

\begin{abstract} \small
Existing item recommendation methods rely on the availability of the user's preferences 
on the items consumed in the past. Additionally, the performance of these recommendation methods
is proportional to the amount of information available to the recommender
system, i.e., the more availability of user's preferences on the items, the  better
the quality of the recommendations for the user.
However, it is not easy to elicit the user's preferences
on items in order to improve the recommendations for the user, particularly when
we have a large number of items. 
In this paper, we propose a method in which we elicit the user's preference on a
\textit{set} of items rather than on the individual items that constitute the
\textit{set}. We use this preference on the set to estimate the user's preference on the items
that constitute the set. This method, compared to the process in which the user
has to sequentially provide his/her preference for the individual items, can
quickly gather the user's preferences on a few sets that span a large number of
items. In the experiments on real world data, we show that the quality of
recommendations generated from these sets is comparable to the recommendations
generated using the individual preference on the items.


\end{abstract}

\section{Introduction}
%!TEX root = paper.tex

Recommender systems are widely deployed in various domains e.g. music (iTunes),
movies (Netflix) and e-commerce (Amazon) to suggest relevant products to the users
based on their preferences. 

One of the key methods used by the recommender systems is the collaborative
filtering method, which uses the information from the previous transactions of 
the users and suggest relevant products to consume in the future. Generally, the collaborative
filtering consists of two approaches: the neighborhood - based approach and the
latent-factor models. In the neighborhood - based approach ~\cite{r30}, the similarities
between the items or the users are used to compute recommendations. While the
latent-factor models ~\cite{r4} transform the users and the items to the same
latent space where both of them are comparable to each other, the items closest
to the users in this space serve as the recommendations to the users.


An important requirement of these collaborative filtering methods is that the
user should indicate his/her preference over a certain number of items to
generate recommendations for future consumption. For existing users this is
not a significant problem, as these users have already consumed some items in the
past and will continue to provide information as they consume the items. 
However, this is a significant problem for the new user as there is no prior 
information available about their preferences and hence the recommender systems
fails to provide personalized recommendations to such users. These users are also
known as "cold-start" users.

To overcome this problem for a new user, most services deploying recommender 
systems elicit a user's preferences on a few popular items which the user may 
have consumed before. 
This process of eliciting the preferences of the users on the items is time
consuming as a user has to indicate his/her preferences on items one at a time.
Hence, the user may quit this preference elicitation process without completing
it, thus leading to poor initial recommendation performance for the user.

In this work, we present a method which elicits preferences of the user on a set of
items rather than rating those items individually in a sequence. This preference
on the set is used to estimate the preferences on the individual items that
constitute the set. Since a set can be composed of multiple items the  user 
can indicate his/her preference on a large
number of items by indicating his/her preference on a few sets of items.  
Hence, the time spent by a cold-start user to elicit his preferences in order to get
personalized recommendations is reduced significantly.


%TODO: Introduce model in simple words, like non-linear and linear and sigmoid add citatons

We make the following contributions: First, we propose various Learning from Set
models that enables estimation of the user's preference on the individual items.
Second, we illustrate the effectiveness of these models with experiments on the 
synthetic and the real datasets by comparing their performance with the state-of-the-art 
matrix factorization methods.


The rest of the paper is organized as follows. Section 2 introduces the
notations used in the paper. In Section 3, the relevant existing methods are
described. Section 4 presents the Learning from Set models. In Section 5,
details about evaluation methodology and dataset are provided. Sections 6
provides the results of the experimental evaluation. Finally, Section 7 gives
some concluding remarks.





\section{Notations and Definitions}
%!TEX root = paper.tex
Throughout the paper, all vectors are column vectors and are represented by
bold lowercase letters (e.g., $\bm{u}$). Matrices are represented by upper
case letters (e.g., ${R,U,V}$).

The historical preference information is represented by a preference matrix $R$.
Each row in ${R}$ corresponds to a user and each column corresponds to an item. 
The entries of $R$ indicates  the users' preferences on the items. 
The preference given by the user $u$ for the item $i$ is represented by entry $r_{u,i}$ in $R$.  
The symbol $\tilde{r}_{u,i}$ represents the score predicted by the model for the actual
preference $r_{u,i}$.

Sets are represented with calligraphic letters. The set of items $\mathcal{S}$
has size $|S|$.





\section{Related Work}
%!TEX root = paper.tex

\subsection{Collaborative Filtering}

Collaborative filtering is one of the widely used methods in the recommender
systems. It tries to estimate the rating on a user $u$ on item $i$ i.e.,
$\tilde{r}_{ui}$  based on the  partially observed user-item rating matrix $R
\in \mathcal{R}^{m \times n}$ for $m$ users and $n$ items. In order to generate
recommendations of new items, we need to estimate the unobserved entries of the
matrix $R$. The unobserved entries can be estimated by assuming the matrix $R$
to be of low-rank and completing the matrix $R$ by minimizing a squared loss:

%TODO: argmin (P,Q) equation, add citations

\begin{equation}
  (\tilde{P}, \tilde{Q}) = \arg \min_{P,Q} \sum_{(u,i)} ( R - [PQ^T]_{u,i})^2,
\end{equation}

where $P \in R^{m \times r}, Q \in R^{n \times r}$. The completed matrix
$\tilde{R} = \tilde{P} \tilde{Q}^T$
is used to serve the recommendation to the user for the items for which his
preferences were unknown in the original matrix $R$.







\section{Methods}
%!TEX root = paper.tex

To estimate the preferences on individual items from the preference of the user on the
sets, we need to understand how the user rate a given set of items. A user may
rate a set of items by considering each item in the set or may chose to rate
the set of the items based on few but majority of the items in the set.

In the first case, when a user consider all the items in the set before 
assigning a rating to a set then we can safely assume that the user most likely
gives the set an average score of his preferences for all the items that constitutes 
the set. Under this assumption the estimated rating of the user $u$ on a 
set $S$ is given by: 

\begin{equation} \label{avgSetEq}
  \begin{split}
    \tilde{r}^{avg}_{us} &= \frac{1}{|S|} \sum_{i \in \mathcal{S}} r_{u,i},
  \end{split}
\end{equation}

where $\mathcal{S}$ denotes the set containing the items and $r_{u,i}$ is the
preference score of the user $u$ on the item $i$ .

Similarly, when a user considers only a few but the majority of the items in
the set $\mathcal{S}$ then the user's estimated rating on the set $\mathcal{S}$ is given by:

\begin{equation} \label{majSetEq}
  \begin{split}
    \tilde{r}^{maj}_{us} &= \frac{1}{|\mathcal{S}_{maj}|} \sum_{i \in
    \mathcal{S}_{maj}} r_{u,i},
  \end{split}
\end{equation}

where $\mathcal{S}_maj$ contains the majority of the top rated items in the set 
$\mathcal{S}$.


Assuming that the original user-item preference matrix $R$ is low-rank, we can write
the estimated preference of the user $u$ for the item $i$ as follow:

\begin{equation} 
  \begin{split}
    \tilde{r}_{ui} &= p_u^Tq_i, 
  \end{split}
\end{equation}

where, $k$ is the rank of the matrix $R$, $p_u \in \mathcal{R}^k$ and $q_i \in \mathcal{R}^K$ denotes the latent factor of the user $u$ and the item $i$ respectively.  

Following the low-rank assumption, we can rewrite the estimated score of the 
user $u$ for a set $\mathcal{S}$ in equations \ref{avgSetEq} and \ref{majSetEq} as follow:

\begin{equation} \label{avgSetLoEq}
  \begin{split}
    \tilde{r}^{avg}_{us} &= \frac{1}{|S|} \sum_{i \in \mathcal{S}} p_u^Tq_i,
  \end{split}
\end{equation}


\begin{equation} \label{majSetLoEq}
  \begin{split}
    \tilde{r}^{maj}_{us} &= \frac{1}{|\mathcal{S}_{maj}|} \sum_{i \in
    \mathcal{S}_{maj}} p_u^Tq_i,
  \end{split}
\end{equation}


\subsection{Learning from Set Model}
The learning from set model is parameterized by $\Theta=[P, Q]$, where the
matrices $P$ and $Q$ contains the latent factors of the users and the items
respectively. The model parameters are estimated by minimizing the squared error
loss function, given by

%
\begin{equation} \label{eq_rmse}
  \mathcal{L}_{rmse}(\Theta) \equiv \sum_{u \in U} \sum_{\substack{s \in
  \mathcal{R}_{us}}} (\tilde{r}_{us} - r_{us})^2,
\end{equation}
%

%TODO: need to add better connection to majority or average assumption

where $\mathcal{R}_{us}$ contains all the sets preferred by the user $u$,
$r_{us}$ is the original rating and $\tilde{r}_{us}$ is the estimated rating of
the user $u$ on the set $s$. The estimated rating $\tilde{r}_{us}$ can be
computed as per equation \ref{avgSetLoEq} or \ref{majSetLoEq} depending on
whether the average or the majority assumption is used to estimate the rating of 
the user $u$ on the set $s$. 

To control model complexity, we add regularization of the model parameters
thereby leading to an optimization process of the following form:

%
\begin{equation} \label{eq_obj}
  \min_{P, Q} \sum_{u \in U} \sum_{\substack{s \in \mathcal{R}_{us}}} (\tilde{r}_{us} - r_{us})^2  + \lambda (\|P\|_F^2 + \|Q\|_F^2),
\end{equation}
%

where $\lambda$ is the regularization parameter.

%TODO: need to add better connection to majority or average assumption

The optimization problem of equation \ref{eq_obj} is solved by stochastic
gradient descent algorithm for the average assumption and stochastic
sub-gradient descent algorithm for the majority assumption.

%TODO: add pseudo-code algorithm
%TODO: add sub-gradient details and why it is needed




\section{Experimental Evaluation}
%\input{experiments.tex}

\section{Results and Discussion}
%\input{results.tex}

\section{Conclusion}
%\input{conclusion.tex}

\bibliography{refs}{}
\bibliographystyle{plain}

\end{document}



