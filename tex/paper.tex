
\documentclass{sig-alternate-05-2015}
\usepackage{amssymb}
\usepackage{amsmath}
\usepackage{bm} 
\usepackage{xspace}
\usepackage{algorithm}
\usepackage{algpseudocode}
\usepackage{varwidth}
\usepackage{threeparttable}
\usepackage{multirow,bigdelim} 
\usepackage{booktabs}
\usepackage{dsfont}
\usepackage{epsfig}
\usepackage{graphicx}
\usepackage{hyperref}
\usepackage{xspace}
\usepackage{color}
\usepackage{epstopdf}
\usepackage[font=footnotesize]{subfig}
\usepackage{chngpage}
\usepackage{subfig}
\usepackage{setspace}




\begin{document}

\title{\Large Learning from Sets}
\author{Mohit Sharma~\thanks{University of Minnesota.}}
\date{}
\maketitle

\begin{abstract} \small
Existing item recommendation methods rely on the availability of the user's preferences 
on the items consumed in the past. Additionally, the performance of these recommendation methods
is proportional to the amount of information available to the recommender
system, i.e., the more availability of user's preferences on the items, the  better
the quality of the recommendations for the user.
However, it is not easy to elicit the user's preferences
on items in order to improve the recommendations for the user, particularly when
we have a large number of items. 
In this paper, we propose a method in which we elicit the user's preference on a
\textit{set} of items rather than on the individual items that constitute the
\textit{set}. We use this preference on the set to estimate the user's preference on the items
that constitute the set. This method, compared to the process in which the user
has to sequentially provide his/her preference for the individual items, can
quickly gather the user's preferences on a few sets that span a large number of
items. In the experiments on real world data, we show that the quality of
recommendations generated from these sets is comparable to the recommendations
generated using the individual preference on the items.


\end{abstract}

\section{Introduction}
%!TEX root = paper.tex

Recommender systems are widely deployed in various domains e.g. music (iTunes),
movies (Netflix) and e-commerce (Amazon) to suggest relevant products to the users
based on their preferences. 

One of the key methods used by the recommender systems is the collaborative
filtering method, which uses the information from the previous transactions of 
the users and suggest relevant products to consume in the future. Generally, the collaborative
filtering consists of two approaches: the neighborhood - based approach and the
latent-factor models. In the neighborhood - based approach ~\cite{r30}, the similarities
between the items or the users are used to compute recommendations. While the
latent-factor models ~\cite{r4} transform the users and the items to the same
latent space where both of them are comparable to each other, the items closest
to the users in this space serve as the recommendations to the users.


An important requirement of these collaborative filtering methods is that the
user should indicate his/her preference over a certain number of items to
generate recommendations for future consumption. For existing users this is
not a significant problem, as these users have already consumed some items in the
past and will continue to provide information as they consume the items. 
However, this is a significant problem for the new user as there is no prior 
information available about their preferences and hence the recommender systems
fails to provide personalized recommendations to such users. These users are also
known as "cold-start" users.

To overcome this problem for a new user, most services deploying recommender 
systems elicit a user's preferences on a few popular items which the user may 
have consumed before. 
This process of eliciting the preferences of the users on the items is time
consuming as a user has to indicate his/her preferences on items one at a time.
Hence, the user may quit this preference elicitation process without completing
it, thus leading to poor initial recommendation performance for the user.

In this work, we present a method which elicits preferences of the user on a set of
items rather than rating those items individually in a sequence. This preference
on the set is used to estimate the preferences on the individual items that
constitute the set. Since a set can be composed of multiple items the  user 
can indicate his/her preference on a large
number of items by indicating his/her preference on a few sets of items.  
Hence, the time spent by a cold-start user to elicit his preferences in order to get
personalized recommendations is reduced significantly.


%TODO: Introduce model in simple words, like non-linear and linear and sigmoid add citatons

We make the following contributions: First, we propose various Learning from Set
models that enables estimation of the user's preference on the individual items.
Second, we illustrate the effectiveness of these models with experiments on the 
real dataset by comparing their performance with the state-of-the-art 
matrix factorization methods.


The rest of the paper is organized as follows. Section 2 introduces the
notations used in the paper. In Section 3, the relevant existing methods are
described. Section 4 presents the Learning from Set models. In Section 5,
details about evaluation methodology and dataset are provided. Sections 6
provides the results of the experimental evaluation. Finally, Section 7 gives
some concluding remarks.





\section{Notations and Definitions}
%!TEX root = paper.tex
Throughout the paper, all vectors are column vectors and are represented by
bold lowercase letters (e.g., $\bm{u}$). Matrices are represented by upper
case letters (e.g., ${R,U,V}$).

The historical preference information is represented by a preference matrix $R$.
Each row in ${R}$ corresponds to a user and each column corresponds to an item. 
The entries of $R$ indicates  the users' preferences on the items. 
The preference given by the user $u$ for the item $i$ is represented by entry $r_{u,i}$ in $R$.  
The symbol $\tilde{r}_{u,i}$ represents the score predicted by the model for the actual
preference $r_{u,i}$.

Sets are represented with calligraphic letters. The set of items $\mathcal{S}$
has size $|S|$.





\section{Related Work}
%!TEX root = paper.tex

\subsection{Collaborative Filtering}

Collaborative filtering is one of the widely used methods in the recommender
systems. It tries to estimate the rating on a user $u$ on item $i$ i.e.,
$\tilde{r}_{ui}$  based on the  partially observed user-item rating matrix $R
\in \mathcal{R}^{m \times n}$ for $m$ users and $n$ items. In order to generate
recommendations of new items, we need to estimate the unobserved entries of the
matrix $R$. The unobserved entries can be estimated by assuming the matrix $R$
to be of low-rank and completing the matrix $R$ by minimizing a squared loss:

%TODO: argmin (P,Q) equation, add citations

\begin{equation}
  (\tilde{P}, \tilde{Q}) = \arg \min_{P,Q} \sum_{(u,i)} ( R - [PQ^T]_{u,i})^2,
\end{equation}

where $P \in R^{m \times r}, Q \in R^{n \times r}$. The completed matrix
$\tilde{R} = \tilde{P} \tilde{Q}^T$
is used to serve the recommendation to the user for the items for which his
preferences were unknown in the original matrix $R$.







\section{Methods}
%!TEX root = paper.tex

\subsection{Modeling rating on the sets}

To estimate the preferences on individual items from the preference of the user on the
sets, we need to understand how the user rate a given set of items. A user may
rate a set of items by considering each item in the set. When a user consider all the 
items in the set before 
assigning a rating to a set then we can safely assume that the user most likely
gives the set an average score of his preferences for all the items that constitute 
the set. This assumption can be validated by the data analysis in the previous
section where we observed that most of the ratings on the sets are close to the
average of the ratings of the items in the set. Under this assumption the  
rating of the user $u$ on a set $S$ is given by: 

\begin{equation} \label{avgSetEq}
  \begin{split}
    r_{us} &= \frac{1}{|S|} \sum_{i \in \mathcal{S}} r_{ui},
  \end{split}
\end{equation}

\noindent where $\mathcal{S}$ denotes the set containing the items and $r_{ui}$ is the
rating of the user $u$ on the item $i$.


Since we do not know the original ratings of the items in the sets, i.e., $r_{ui}$
we can estimate the rating on the set, i.e., $\tilde{r}_{us}$ as the average of the estimated ratings of
the items in the set.

\begin{equation} \label{avgSetEstEq}
  \begin{split}
    \tilde{r}_{us} &= \frac{1}{|S|} \sum_{i \in \mathcal{S}} \tilde{r}_{ui},
  \end{split}
\end{equation}

\noindent where $\tilde{r}_{ui}$ is the estimated rating of the user $u$ on the
item $i$.


\subsection{Modeling user-item ratings}

Assuming that the original user-item rating matrix $R$ is low-rank, then
following the classic matrix factorization approach~\cite{r43} the estimated
rating of the user $u$ for the item $i$ is given by,

\begin{equation} \label{biasRatPredEq}
  \begin{split}
    \tilde{r}_{ui} &= b_u + b_i + p_u^Tq_i, 
  \end{split}
\end{equation}


\noindent where $b_u$ is the user bias, $b_i$ is the item bias, $p_u \in \mathcal{R}^k$ denotes the latent factor of the user
$u$, $q_i \in \mathcal{R}^K$ denotes the latent factor of the item $i$ and $k$ 
is the rank of the matrix $R$.  

\subsection{Set rating using matrix-factorization (LFS)}
We can rewrite the estimated score of the user $u$ for the set $\mathcal{S}$ 
in equation~\ref{avgSetEstEq} using equation~\ref{biasRatPredEq} as follow:

\begin{equation} \label{avgSetLoEq}
  \begin{split}
    \tilde{r}_{us} &= \frac{1}{|S|} \sum_{i \in \mathcal{S}} b_u + b_i + p_u^Tq_i,
  \end{split}
\end{equation}

\subsection{Modeling session bias (LFSWSessBias)}
Further, the user's ratings on the set could be affected by psychological
phenomena or his mood during the session, e.g., a user may
rate a set in the context of the sets they have seen before. Also, as seen in our
investigations on the data, some users tend to overrate or underrate the sets
when compared with the average of the ratings of the items in the set.
Such effects can be captured by adding a user specific session bias:

\begin{equation} \label{avgSetWSessBiasEq}
  \begin{split}
    \tilde{r}_{us} &= b_{us} + \frac{1}{|S|} \sum_{i \in \mathcal{S}} b_u + b_i + p_u^Tq_i,
  \end{split}
\end{equation}

\noindent where $b_{us}$ denotes the session bias of the user $u$.


\subsection{Full bias model (LFSWGBias)}
Moreover, We can also add the mean of ratings on the sets (a constant) to represent  
the portion of the rating on the set which is independent of the users' and the
items' personalization:

\begin{equation} \label{avgSetWGBiasEq}
  \begin{split}
    \tilde{r}_{us} &= \mu + b_{us} + \frac{1}{|S|} \sum_{i \in \mathcal{S}} b_u + b_i + p_u^Tq_i,
  \end{split}
\end{equation}

\noindent where $\mu$ is the mean of ratings on the sets.

\subsection{Model learning}
The model parameters, i.e., $\theta= [p_u, q_u, b_u, b_i, b_{us}]$, are estimated 
by minimizing a loss function. 
For accurate predictions of ratings for the sets and the items, it is
appropriate to minimize a square error loss function to estimate the model. In
the Top-$n$  recommendations, the precise rating prediction is irrelevant as we
only care about correctly ranking the items for the user. Therefore, a ranking
loss function, e.g., Bayesian Personalized Ranking (BPR)~\cite{r6} is more appropriate to
estimate the model. We propose to use both the square error loss function and
the BPR loss function to estimate the model parameters.

The loss function to estimate the model by minimizing square error loss function
is given by
 
%
\begin{equation} \label{eq_rmse}
  \mathcal{L}_{rmse}(\Theta) \equiv \sum_{u \in U} \sum_{\substack{s \in
  \mathcal{R}_{us}}} (\tilde{r}_{us}(\Theta) - r_{us})^2,
\end{equation}
%


where $U$ represents all the users, $\mathcal{R}_{us}$ contains all the sets rated 
by the user $u$, $r_{us}$ is the original rating of the user $u$ on the set $s$ 
and $\tilde{r}_{us}$ is the estimated rating of the user $u$ on the set $s$. 


Similarly, the model can be learned by minimizing the BPR loss function given by

\begin{equation} \label{eq_bpr}
\mathcal{L}_{bpr}(\Theta) \equiv - \sum_{u \in U} \sum_{\substack{s,t \in
\mathcal{R}_{us} ,\\  r_{us} > r_{ut}}}  \ln \, \sigma(\tilde{r}_{us}(\Theta) -
\tilde{r}_{ut}(\Theta) ),
\end{equation}

\noindent where $s$ and $t$ are sets rated by user $u$ such that $r_{us} >
r_{ut}$.


To control model complexity, we add regularization of the model parameters
thereby leading to an optimization process of the following form:

%
\begin{equation} \label{eq_obj}
  \min_{\Theta} \mathcal{L}(\Theta)  + \lambda (||\Theta||^2),
\end{equation}

%
\noindent where $\lambda$ is the regularization parameter and
$\mathcal{L}(\theta)$ is the loss function, i.e., either $\mathcal{L}_{bpr}(\theta)$ or
$\mathcal{L}_{rmse}(\theta)$.

%TODO: need to add better connection to majority or average assumption

The optimization problem of the equation \ref{eq_obj} can be solved by stochastic
gradient descent algorithm.


\begin{algorithm}[t]
  \caption{$LFS_{rmse}$-Learn}
  \label{alg-lfs-rmse}
  \begin{algorithmic}[1]
    \Procedure{LFS$_{rmse}$-Learn}{}
      \State $\eta \gets$ learning rate
      \State $\lambda \gets$ regularization weights
      \State iter $\gets$ 0
      \State Initialize $\Theta$ randomly
      \While {iter $<$ maxIter or error on validation set decreases}
        \For{each user $u$}
          \State Sample a set $s$ s.t. $s \in \mathcal{R}_{us}$ 
          \State Compute $\tilde{r}_{us} $ using equation~\ref{avgSetWGBiasEq}
          \State $e_{us} \gets (\tilde{r}_{us} - r_{us})$
          \State $v_k \in \mathcal{R}^k$  $\gets 0$
          \For {each item $i \in s$ }
            \State $v_k \gets v_k + q_i$
          \EndFor
          \State $p_u \gets p_u - \eta(\frac{e_{us}}{|s|} v_k + \lambda p_u)$
          \State $b_u \gets b_u - \eta(e_{us} + \lambda b_u)$
          \State $b_{us} \gets b_{us} - \eta(e_{us} + \lambda b_{us})$
          \For {each item $i \in s$ }
            \State $q_i \gets q_i - \eta(\frac{e_{us}}{|s|} p_u + \lambda q_i)$  
            \State $b_i \gets b_i - \eta(\frac{e_{us}}{|s|} + \lambda b_i)$
          \EndFor
        \EndFor
        \State iter $\gets $ iter $ + 1$
      \EndWhile
      \State \Return $\Theta$
    \EndProcedure
  \end{algorithmic}
\end{algorithm}


\begin{algorithm}[t]
  \caption{$LFS_{bpr}$-Learn}
  \label{alg-lfs-bpr}
  \begin{algorithmic}[1]
    \Procedure{LFS$_{bpr}$-Learn}{}
      \State $\eta \gets$ learning rate
      \State $\lambda \gets$ regularization weights
      \State iter $\gets$ 0
      \State Initialize $\Theta$ randomly
      \While {iter $<$ maxIter or error on validation set decreases}
        \For{each user $u$}
        \State Sample  a pair of set $s,t \in \mathcal{R}_{us}$ s.t. $r_{us} < r_{ut}$
        \State Compute $\tilde{r}_{us} $ and $\tilde{r}_{ut}$ using
        equation~\ref{avgSetLoEq}
        \State $\tilde{r}_{ust} \gets \tilde{r}_{us} - \tilde{r}_{ut}$
        \State $\tau \gets \frac{-1}{1 + exp(\tilde{r}_{ust})}$
          \State $v_k \in \mathcal{R}^k$  $\gets 0$
          \For {each item $i \in s$ }
            \State $v_k \gets v_k + q_i$
          \EndFor
          \For {each item $i \in t$ }
            \State $v_k \gets v_k - q_i$
          \EndFor
          \State $p_u \gets p_u - \eta(\frac{\tau}{|s|} v_k + \lambda p_u)$
          \For {each item $i \in s$ }
            \State $q_i \gets q_i - \eta(\frac{\tau}{|s|} p_u + \lambda q_i)$  
            \State $b_i \gets b_i - \eta(\frac{\tau}{|s|} + \lambda b_i)$
          \EndFor
          \For {each item $i \in t$ }
            \State $q_i \gets q_i - \eta(\frac{-\tau}{|s|} p_u + \lambda q_i)$  
            \State $b_i \gets b_i - \eta(\frac{-\tau}{|s|} + \lambda b_i)$
          \EndFor
        \EndFor
        \State iter $\gets $ iter $ + 1$
      \EndWhile
      \State \Return $\Theta$
    \EndProcedure
  \end{algorithmic}
\end{algorithm}



\iffalse
The model parameters can also be estimated by minimizing a pair-wise loss
function. In the pair-wise loss, we are not interested in the absolute ratings
of the set but rather in the difference or relative ordering of the preference
over the sets. Let $s$ and $t$ be two sets rated by a user $u$ such that $r_{us}
> r_{ut}$, then the estimated ratings on the set,i.e., $\tilde{r_{us}}$ and
$\tilde{r_{ut}}$ should preserve the relative order and difference in their
values. A simple loss to consider in ranking is the number of pairs incorrectly 
ordered by the ranking model. This loss is referred as the zero-one loss,

\begin{equation} \label{eq_01}
  \mathcal{E}_{01}(\Theta) \equiv \sum_{u \in U} \sum_{\substack{s,t \in
\mathcal{R}_{us},\\ r_{us} > r_{ut}}} \bm{1}(r_{ust} . \tilde{r_{ust}} < 0),
\end{equation}

\noindent where $r_{ust} = r_{us} - r_{ut}$, $\tilde{r_{ust}} = \tilde{r_{us}} -
\tilde{r_{ut}}$ and $\bm{1}(x)$ is the indicator function. Since, the zero-one
loss in equation~\ref{eq_01} is not differentiable we use a smooth surrogate
loss function that forms a convex upper bound on the zero-one loss function.


\begin{equation} \label{eq_smooth}
  \mathcal{E}(\Theta) \equiv \sum_{u \in U} \sum_{\substack{s,t \in
  \mathcal{R}_{us},\\ r_{us} > r_{ut}}} \mathcal{L}_{surr}(r_{ust}, \tilde{r_{ust}}),
\end{equation}

\noindent where $\mathcal{L}_{surr}(x,y)$ is a convex loss function. We used
$\mathcal{L}_{Hinge}(x,y)=[\gamma + x - y]_+$ and $\mathcal{L}_{Log}(x,y) =
log(1+exp(\gamma + x - y))$ as the pairwise loss functions, where $\gamma$ is a
free parameter.
\fi

%TODO: add pseudo-code algorithm




\section{Experimental Evaluation}
%\input{experiments.tex}

\section{Results and Discussion}
%\input{results.tex}

\section{Conclusion}
%\input{conclusion.tex}

\bibliography{refs}{}
\bibliographystyle{plain}

\end{document}



