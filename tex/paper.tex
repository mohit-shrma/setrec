


\usepackage{sig-alternate-05-2015}


\begin{document}

\title{\Large Learning from Sets}
\author{Mohit Sharma~\thanks{University of Minnesota.}}
\date{}
\maketitle

\begin{abstract} \small
Existing item recommendation methods rely on a user's preferences on the items
consumed in the past. However, it is not easy to elicit the user's preferences
on each and every item to improve the recommendation to user, particularly when
we have a large number of items. In this paper, we present a method to elicit
the user's preference on a set of items rather than on the items that constitute
the set and use this preference on the set to estimate the preference of the
constituting items. This method, compared to the process in which the user has
to provide his/her preference for the individual items, can quickly gather the
user's preferences on a few sets that span a large number of items. In the
experiments on the real world data, we show that the estimated preferences on
the individual items from these sets are comparable to the preferences when
estimated using the individual preference on the items. 


\end{abstract}

\section{Introduction}
%!TEX root = paper.tex

Recommender systems are widely deployed in various domains e.g. music (iTunes),
movies (Netflix) and e-commerce (Amazon) to suggest relevant products to the users
based on their preferences. 

One of the key methods used by the recommender systems is the collaborative
filtering method, which uses the information from the previous transactions of 
the users and suggest relevant products to consume in the future. Generally, the collaborative
filtering consists of two approaches: the neighborhood - based approach and the
latent-factor models. In the neighborhood - based approach ~\cite{r30}, the similarities
between the items or the users are used to compute recommendations. While the
latent-factor models ~\cite{r4} transform the users and the items to the same
latent space where both of them are comparable to each other, the items closest
to the users in this space serve as the recommendations to the users.


An important requirement of these collaborative filtering methods is that the
user should indicate his/her preference over a certain number of items to
generate recommendations for future consumption. For existing users this is
not a significant problem, as these users have already consumed some items in the
past and will continue to provide information as they consume the items. 
However, this is a significant problem for the new user as there is no prior 
information available about their preferences and hence the recommender systems
fails to provide personalized recommendations to such users. These users are also
known as "cold-start" users.

To overcome this problem for a new user, most services deploying recommender 
systems elicit a user's preferences on a few popular items which the user may 
have consumed before. 
This process of eliciting the preferences of the users on the items is time
consuming as a user has to indicate his/her preferences on items one at a time.
Hence, the user may quit this preference elicitation process without completing
it, thus leading to poor initial recommendation performance for the user.

In this work, we present a method which elicits preferences of the user on a set of
items rather than rating those items individually in a sequence. This preference
on the set is used to estimate the preferences on the individual items that
constitute the set. Since a set can be composed of multiple items the  user 
can indicate his/her preference on a large
number of items by indicating his/her preference on a few sets of items.  
Hence, the time spent by a cold-start user to elicit his preferences in order to get
personalized recommendations is reduced significantly.


%TODO: Introduce model in simple words, like non-linear and linear and sigmoid add citatons

We make the following contributions: First, we propose various Learning from Set
models that enables estimation of the user's preference on the individual items.
Second, we illustrate the effectiveness of these models with experiments on the 
synthetic and the real datasets by comparing their performance with the state-of-the-art 
matrix factorization methods.


The rest of the paper is organized as follows. Section 2 introduces the
notations used in the paper. In Section 3, the relevant existing methods are
described. Section 4 presents the Learning from Set models. In Section 5,
details about evaluation methodology and dataset are provided. Sections 6
provides the results of the experimental evaluation. Finally, Section 7 gives
some concluding remarks.





\section{Notations and Definitions}
%!TEX root = paper.tex
Throughout the paper, all vectors are column vectors and are represented by
bold lowercase letters (e.g., $\bm{u}$). Matrices are represented by upper
case letters (e.g., ${R,U,V}$).

The historical preference information is represented by a preference matrix $R$.
Each row in ${R}$ corresponds to a user and each column corresponds to an item. 
The entries of $R$ indicates  the users' preferences on the items. 
The preference given by the user $u$ for the item $i$ is represented by entry $r_{u,i}$ in $R$.  
The symbol $\tilde{r}_{u,i}$ represents the score predicted by the model for the actual
preference $r_{u,i}$.

Sets are represented with calligraphic letters. The set of items $\mathcal{S}$
has size $|S|$.





\section{Related Work}
%!TEX root = paper.tex

\subsection{Collaborative Filtering}

Collaborative filtering is one of the widely used methods in the recommender
systems. It tries to estimate the rating on a user $u$ on item $i$ i.e.,
$\tilde{r}_{ui}$  based on the  partially observed user-item rating matrix $R
\in \mathcal{R}^{m \times n}$ for $m$ users and $n$ items. In order to generate
recommendations of new items, we need to estimate the unobserved entries of the
matrix $R$. The unobserved entries can be estimated by assuming the matrix $R$
to be of low-rank and completing the matrix $R$ by minimizing a squared loss:

%TODO: argmin (P,Q) equation, add citations

\begin{equation}
  (\tilde{P}, \tilde{Q}) = \arg \min_{P,Q} \sum_{(u,i)} ( R - [PQ^T]_{u,i})^2,
\end{equation}

where $P \in R^{m \times r}, Q \in R^{n \times r}$. The completed matrix
$\tilde{R} = \tilde{P} \tilde{Q}^T$
is used to serve the recommendation to the user for the items for which his
preferences were unknown in the original matrix $R$.







\section{Methods}







\section{Experimental Evaluation}
\input{experiments.tex}

\section{Results and Discussion}
\input{results.tex}

\section{Conclusion}
\input{conclusion.tex}

\bibliography{refs}{}
\bibliographystyle{plain}

\end{document}



